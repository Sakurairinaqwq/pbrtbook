\section{习题}\label{sec:习题06}
\begin{enumerate}
      \item \circletwo 一些类型的相机通过在胶片上滑动矩形狭缝
      来曝光胶片。当物体移动方向与曝光狭缝不同时
      会产生有趣效应\citep{761554,10.1080/2151237X.2007.10129235}。
      而且,大多数数字相机在几毫秒时段内从连续扫描线中读出像素值;
      这会造成具有相似视觉特性的\keyindex{卷帘快门}{rolling shutter}{}痕迹。
      在本章的一个或多个相机实现中修改生成时间样本的方式对该效应建模。
      渲染能清楚展示考虑该问题所带来的影响的运动物体图像。
      \item \circletwo 编写一个应用加载由\refvar{EnvironmentCamera}{}渲染的图像,
      并利用纹理映射将其应用在球心位于视点处的球体上,
      使得可以交互地观察它们。用户应能够自由改变观察方向。
      如果为生成球上的纹理坐标使用了正确的纹理映射函数,
      则该应用生成的图像看起来像是观察者处于渲染时相机在场景中的位置,
      因此给予用户交互地环顾场景的能力。
      \item \circletwo \refvar{RealisticCamera}{}中的光圈
      建模为完美的圆;对于可调光圈相机,其光圈通常由
      直边可动叶片构成,因此是$n$边形。修改\refvar{RealisticCamera}{}以
      建模更真实的光圈形状并渲染能展示你模型差异的图像。
      你可能会发现渲染小、亮、失焦物体的场景(例如镜面高光)
      对展示其差异很有用。
      \item \circletwo 计算机图形学中景深的标准模型把
      弥散圆建模为将场景中一点成像为均匀强度的圆盘,
      然而许多真实透镜产生的弥散圆有非线性的变化,例如高斯分布。
      该效应称为\keyindex{散景}{bokeh}{}\citep{10.1145/1242073.1242155}。
      例如,当失焦时观察小光点,反射折射式\sidenote{译者注:原文catadioptric。}(镜面)
      透镜会产生环形高光。修改\refvar{RealisticCamera}{}中
      景深的实现(例如通过偏置透镜样本位置的分布)以生成具有该效应的图像。
      渲染能展示其与标准模型间差异的图像。
      \item \circletwo \keyindex{焦点堆栈渲染}{focal stack rendering}{}:
      焦点堆栈是一固定场景的一系列图像,每张图像的相机对焦距离不同。
      \citet{5740919}与\citet{Jacobs2012}介绍了焦点堆栈的许多应用,
      包括任意的景深,即用户可以指定任意深度对焦,达到传统光学不可能有的效果。
      用pbrt渲染焦点堆栈并编写交互工具以用其控制对焦效应。
      \item \circlethree \keyindex{光场相机}{light field camera}{camera相机}:
      \citet{ng:hal-02551481}讨论了一种相机的物理设计和应用,
      它能捕获胶片上各出射光瞳的微小图像,而不是像常规相机那样
      对每一像素在整个出射光瞳上的辐射求平均。这样的相机会获取光场的表示——
      到达相机传感器的辐射在空间和方向上变化着的分布。
      通过获取光场,就能实现许多有趣操作,包括拍摄相片后重新对焦。
      阅读\citeauthor{ng:hal-02551481}的论文并在pbrt中实现一个
      获取场景光场的\refvar{Camera}{}。编写工具以允许用户交互地重新对焦这些光场。
      \item \circlethree \refvar{RealisticCamera}{}的实现将胶片置于
      光轴中心并与之垂直。尽管这是真实相机最常见的构造,但通过调整胶片
      相对于透镜系统的放置可以得到有趣的效应。例如,当前实现的焦平面
      总是垂直于光轴;如果胶片平面(或透镜系统)倾斜使得胶片不垂直于光轴,
      则焦平面不再垂直于光轴(这对风景摄影很有用,例如让焦平面
      和地平面对齐时甚至用更大光圈也能有更大景深)。或者可以平移胶片平面
      使得它不在光轴中心;例如可利用该平移保持焦平面和极高物体对齐。
      修改\refvar{RealisticCamera}{}以允许这一或两种调整并渲染展示结果的图像。
      注意当前实现的许多地方(例如出射光瞳计算)都有这些修改会违背的假设需要你解决。
\end{enumerate}