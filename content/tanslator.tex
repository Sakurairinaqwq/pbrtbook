\chapter{译者序}\label{chap:译者序}

“为什么要翻译这本书?”

——这是自我翻译本书以来被问过的最多的问题。
同学问过、父母问过、连面试官也问过。
我回答过各种各样的理由,
甚至连自己也不知道哪个理由才是最重要最关键的。

我曾经有过比较低谷的学业经历,
这主要是心中的完美主义作祟。
当主业不能使自己获得一丝满足时,
我不由得把目光投向了其他领域——渲染便是其中之一。
之后很快我便在GitHub上偶遇了pbrt-v4项目。
虽然这并不是我第一次与渲染相遇了——

若干年前读初中的时候,家里终于有了第一台电脑。
那个时候老家电脑并不算普及。爸爸为了学会怎么用
还得去书店买本《电脑入门基础教程》之类的书回来查阅。
我在学校微机课倒是学会了开关机和使用开始菜单,所以对这些书已没多大兴趣。
不过我在这些教程书区中发现了其他宝贝——与电脑相关的视觉设计类书籍,
就是那些封面图片非常有“质感”的与渲染相关的书,厚度通常也很离谱。
同龄人如果家里新买了电脑,几乎都会兴奋地琢磨怎么安装上时下最火的游戏好好玩一把,
而我却被这些书“带偏”了——虽然完全看不懂其中的内容,甚至不知道
里面偶尔出现的单词“run”是“运行”的意思而不是“奔跑”,
但书中花花绿绿的模型和最终展现的成品渲染图深深地吸引了我。
在电脑里生成一幅现实生活中不存在但分毫毕现的图像,
这样的技术对于一个刚刚学会有理数四则运算的学生而言是非常震撼的。
我让爸爸买回其中一本,回去对照着折腾起怎么安装3ds Max\textsuperscript{\textregistered}
和V-Ray\textsuperscript{\textregistered},
然后磕磕绊绊地按书中步骤设置材质并花三个多小时渲染出一幅室内装修图。
现在想来,当年那台连独显都没有的电脑承受了太多不该它承受的计算开销。
我也只渲染过那么一次——模型、材质都是附带光盘里的,
照着做了一遍后我也学不会什么,只是觉得好玩。
也许那时候对渲染的兴趣种子就这么埋进心里了。

直到我与它再会时,这些回忆便苏醒过来。
那时神经网络的黑箱特性让我厌倦(这大概只是我的偏见),
而渲染技术明晰的数学原理如同命中十环那般精准地满足了我的口味,
让我有一种从迷茫中解脱的释然感。
事实上,我一直有完成一部“作品”的愿望。
在我心里“作品”这个词是有很高门槛的。
我发表了主业相关的论文,但我并不喜欢它——那远远算不上“作品”。
作为骨灰级动漫爱好者,我甚至觉得自己倾注情感剪辑
的MAD(对动画原片进行重新剪辑配乐做成的视频)更算得上“作品”。
何况动漫粉丝的身份让我对渲染技术的滤镜又加深了一层。
在确认pbrt是一套可以自学搞定的完整教程后,
我认定这本内容详实的著作就是我心中追求的“作品”的模样。
我下定决心以公开翻译的方式学习它,也许要很久,但不怕学不懂。
毕竟检验是否学会的最好办法就是看能否教会别人。
毫不避讳地说,我就是希望从这个过程中获取自我认同,满足完美主义心理。

这本书翻译起来确实不轻松,曾经我只能抽课余时间写,现在只能抽业余时间写。
越写就越觉得原作者能把如此丰富的内容无偿公开是何其慷慨,
所以我效仿着不对获取译本设任何门槛——而且许可证也不允许。
知识本应自由而充分地交流。我崇尚开源精神,并通过这种方式为其贡献力量。
比起从中获取什么经济利益,我更看中是否有人因此而顺利入门渲染技术,成为又一个充满潜力的领域新人。
当然坦诚地说,我没有从翻译本书中获得个人利益是不可能的——
我收获了一些零散的朋友圈点赞满足了自己的虚荣心,
还把翻译经历写进简历里帮助自己熬过了求职关卡。
这也算是对自己一点小小的犒劳吧。

最后是一点关于匿名发表的解释。
“为什么不用真名而是用笔名Kanition署名译作?”
“对你找工作有什么用吗?”
“不怕别人冒充你去攫取个人名利吗?”
“花了这么多心血难道不希望自己的名字在业内传开吗?”
我当然想!可是我起初并不知道自己能翻译到什么水平。
要知道当初我读第一章面对那么多陌生概念是非常痛苦的。
与其搞砸后被人拿去嘲笑黑历史,不如留下一丝神秘感,
像江湖上不见其人的侠者一般只留下一个名号。

此外,这个笔名对我而言有特殊的意义。
它改编自国外某家我十分钟爱的动画公司名。
在我心中,这家公司是耕耘“作品”的代表,
正是她制作的动画让我撑过翻译本书之前那段煎熬的岁月。
然而一场毫无道理的恐怖袭击突然夺走了这些鲜活的生命,那些作品成为了永恒的遗作……
几年过去,我很高兴看到她能重焕活力,以不输曾经的水准再次为观众奉上一部部令人难忘的“作品”。
着手翻译后不久,我又搜索到了某著名游戏工作室
已故前辈{\href{https://github.com/QianMo}{\textcolor{black}{\fbox{毛星云}}}}在网上无私分享的大量渲染技术内容。
尽管我未曾拜读全文,但只是简单浏览也能感觉得到他投入了多少心血。
听闻他逝世前正在翻译另一本渲染巨著——第4版《\citetitle{978-1138627000}》。
如果他没有发生悲剧,现在译本应该已经面世了吧。
从某种意义上说,我们在做一样的事情。

总之,无论是渲染还是动画或游戏制作,我们都是在构建一个更美丽纯粹的世界。
Kanition这个名字正是代表着为此努力的人。
所以即便这本译作将来获得好评,我也不会改回真名署名,
这是我为数不多纪念他们的方式了。

\vspace{15pt}
{\hfill {\itshape 译者}\quad\raisebox{-0.2\height}{{$\includegraphics[height=20pt]{Pictures/preface/sign.eps}$}}\qquad}

\section*{致谢}\label{sec:致谢b}
本书\LaTeX 模板“\href{https://www.latextemplates.com/template/legrand-orange-book}{The Legrand Orange Book}”
由\href{mailto:legrand.mathias@gmail.com}{Mathias Legrand}和\href{mailto:vel@latextemplates.com}{Vel}开发,
遵循\href{https://creativecommons.org/licenses/by-nc-sa/3.0/}{署名-非商业性使用-相同方式共享3.0(CC BY-NC-SA 3.0)}许可。

感谢\href{https://zixuan-zhang.com}{Zixuan Zhang}参与翻译\refsec{相机模型}部分段落;
感谢\href{https://github.com/xwc2021}{xwc2021}对\refsub{辐射度量}一节的讨论;
感谢\href{https://github.com/OverflowCat}{OverflowCat}、
\href{https://www.zhihu.com/people/zhong-ling-xiao}{钟凌霄}、
\href{https://github.com/star-hengxing}{star-hengxing}、
\href{https://theigrams.github.io/}{张晋}、
\href{https://github.com/leemeans}{leemeans}、
\href{https://www.zhihu.com/people/itact/posts}{DeJhon-Huang}、
\href{https://github.com/CellarCordial}{CellarCordial}、
\href{https://github.com/LeonHaoWI}{LeonHaoWI}
改进部分词句。