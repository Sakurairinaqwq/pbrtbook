\section{曲线}\label{sec:曲线}
\begin{remark}
    本节含有高级内容,第一次阅读时可以跳过。
\end{remark}

虽然三角形可用于表示细小形状来建模细微几何体如头发、皮毛或草地,
但为了更高效地渲染这类物体,拥有专门的\refvar{Shape}{}是值得的,
因为经常出现它们的许多实例。
本节介绍的\refvar{Curve}{}形状即表示
用\keyindex{三次贝塞尔样条}{cubic Bézier spline}{spline样条}建模的细小几何体,
它由四个控制点$\bm p_0,\bm p_1,\bm p_2$和$\bm p_3$定义。
贝塞尔样条穿过第一个和最后一个控制点;
中间点由多项式给出(见\reffig{3.15})
\begin{align}\label{eq:3.3}
    \bm p(u)=(1-u)^3\bm p_0+3(1-u)^2u\bm p_1+3(1-u)u^2\bm p_2+u^3\bm p_3\, .
\end{align}
给定在其他三次基中指定的曲线,例如\keyindex{Hermite样条}{Hermite spline}{spline样条}
\sidenote{译者注:三次Hermite样条
    形如$\bm p(u)=(2u^3-3u^2+1)\bm p_0+(u^3-2u^2+t)\bm m_0+(-2u^3+3u^2)\bm p_1+(u^3-u^2)\bm m_1$,
    其中$\bm p_i$是起点和终点,$\bm m_i$是起点和终点处的切向量,$u\in[0,1]$.},
很容易将其转化为贝塞尔基,所以这里的实现将这一负担留给用户。
如果经常用的话,可以很容易添加该功能。
\begin{figure}[htbp]
    \centering\input{Pictures/chap03/bezier-and-cps.tex}
    \caption{三次贝塞尔曲线由四个控制点$\bm p_i$定义。
        \protect\refeq{3.3}定义的曲线$\bm p(u)$分别在$u=0$和$u=1$处穿过第一个和最后一个控制点。}
    \label{fig:3.15}
\end{figure}

\refvar{Curve}{}形状由具有宽度的1D贝塞尔曲线定义,
其宽度是从起点宽度到终点宽度沿其范围线性插值的。
即它们定义了一个平坦的2D曲面(\reffig{3.16})
\footnote{注意术语的滥用:尽管曲线是1D数学概念,但\protect\refvar{Curve}{}形状表示2D曲面。
    下文中我们所称的曲线通常指\protect\refvar{Shape}{}。
    当区别不够明确时我们用“贝塞尔曲线”来区分1D概念。}。
直接用光线与该表示相交而不用细化\sidenote{译者注:原文tessellating。}它是可能的,
从而能够高效渲染平滑曲线而不使用过多内存。
\begin{figure}[htbp]
    \centering\input{Pictures/chap03/bezier-and-offsets.tex}
    \caption{\protect\refvar{Curve}{}形状的基本几何。
        1D贝塞尔曲线在沿它的每一点上都朝两边垂直于曲线的方向偏移指定宽度的一半。
        得到的区域表示该曲线的曲面。}
    \label{fig:3.16}
\end{figure}

\reffig{3.17}展示了用上百万\refvar{Curve}{}建模皮毛的小兔子模型。
\begin{figure}[htbp]
    \centering\includegraphics[width=\linewidth]{chap03/furrybunny.png}
    \caption{毛茸茸的小兔子。上百万\refvar{Curve}{}形状用于建模皮毛的小兔子模型。
        这里我们使用了不真实的长曲线来更好展示\refvar{Curve}{}的能力。}
    \label{fig:3.17}
\end{figure}
\begin{lstlisting}
`\initcode{CurveDeclarations}{=}`
class `\initvar{Curve}{}` : public `\refvar{Shape}{}` {
public:
    `\refcode{Curve Public Methods}{}`
private:
    `\refcode{Curve Private Methods}{}`
    `\refcode{Curve Private Data}{}`
};
\end{lstlisting}
\begin{lstlisting}
`\initcode{Curve Private Methods}{=}`
bool `\refvar{recursiveIntersect}{}`(const `\refvar{Ray}{}` &r, `\refvar{Float}{}` *tHit,
    `\refvar{SurfaceInteraction}{}` *isect, const `\refvar{Point3f}{}` cp[4],
    const `\refvar{Transform}{}` &rayToObject, `\refvar{Float}{}` u0, `\refvar{Float}{}` u1, int depth) const;

\end{lstlisting}

如\reffig{3.18}所示,\refvar{Curve}{}形状可以表示三种曲线。
\begin{itemize}
    \item \keyindex{平坦面}{flat}{}:这种表示的曲线总是朝向与之相交的光线;
          它们在建模细微扫掠圆柱形状如头发或皮毛时很有用。
    \item {\sffamily 圆柱}:对于在屏幕上张开几像素的曲线(如不远处看到的意大利面),
          \refvar{Curve}{}形状可以计算着色法线使得曲线看起来实际是圆柱体。
    \item \keyindex{丝带}{ribbon}{}:这个变体对于建模实际没有
          圆柱\keyindex{横截面}{cross section}{}的形状(例如一片草)很有用。
\end{itemize}
\begin{figure}[htbp]
    \centering\includegraphics[width=0.5\linewidth]{chap03/threecurves.png}
    \caption{\protect\refvar{Curve}{}形状可以表示的三种曲线。
        左边是平坦曲线,总是朝向垂直于光线接近它的方向。
        中间的变体可以设置着色法线使得曲线看起来是圆柱体。
        右边是丝带,在起点和终点有固定朝向;中间的朝向在它们之间平滑插值。}
    \label{fig:3.18}
\end{figure}

枚举\refvar{CurveType}{}记录了给定的\refvar{Curve}{}实例模型是它们中的哪一个。

平坦和圆柱曲线变体旨在用于方便地近似变形的圆柱体。
要注意的是针对它们求得的相交并不对应物理可实现的3D形状,
在拿真正的圆柱体场景作参考时这可能导致微小的矛盾。
\begin{lstlisting}
`\initcode{CurveType Declarations}{=}`
enum class `\initvar{CurveType}{}` { `\initvar{Flat}{}`, `\initvar[CurveType::Cylinder]{Cylinder}{}`, `\initvar{Ribbon}{}` };
\end{lstlisting}

给定在pbrt场景描述文件中指定的曲线,
值得将其分为若干段,每段包含曲线的一部分参数$u$范围。
(这样做的一个原因是轴对齐边界框不能紧致地包围起伏的曲线,
但把贝塞尔样条分开会使其不那么弯——
多项式样条的\keyindex{变差缩减性}{variation diminishing property}{}。)
因此,\refvar{Curve}{}构造函数同时
接收$u$值的参数范围$[u_{\min},u_{\max}]$以及
指向结构体\refvar{CurveCommon}{}的指针,
它存储了控制点和各曲线段共享的关于曲线的其他信息。
这样,单个曲线段所占内存被最小化,
使得在内存中保留许多曲线更容易。
\begin{lstlisting}
`\initcode{Curve Public Methods}{=}`
`\refvar{Curve}{}`(const `\refvar{Transform}{}` *ObjectToWorld, const `\refvar{Transform}{}` *WorldToObject,
      bool reverseOrientation, const std::shared_ptr<`\refvar{CurveCommon}{}`> &common,
      `\refvar{Float}{}` uMin, `\refvar{Float}{}` uMax)
    : `\refvar{Shape}{}`(ObjectToWorld, WorldToObject, reverseOrientation),
      `\refvar{common}{}`(common), `\refvar{uMin}{}`(uMin), `\refvar{uMax}{}`(uMax) { }
\end{lstlisting}
\begin{lstlisting}
`\initcode{Curve Private Data}{=}`
const std::shared_ptr<`\refvar{CurveCommon}{}`> `\initvar{common}{}`;
const `\refvar{Float}{}` `\initvar{uMin}{}`, `\initvar{uMax}{}`; 
\end{lstlisting}

\refvar{CurveCommon}{}构造函数大多数只初始化成员变量
和传入的控制点值、曲线宽度等。
提供给它的控制点应该在曲线的物体空间内。

对于\refvar{Ribbon}{}曲线,\refvar{CurveCommon}{}存储了
曲面法线以在每个终点处让曲线定向。
构造函数预计算两个法向量的夹角并求该角正弦的倒数;
这些值会在沿曲线范围计算任意点处的朝向时很有用。
\begin{lstlisting}
`\initcode{Curve Method Definitions}{=}\initnext{CurveMethodDefinitions}`
`\refvar{CurveCommon}{}`::`\refvar{CurveCommon}{}`(const `\refvar{Point3f}{}` c[4], `\refvar{Float}{}` width0, `\refvar{Float}{}` width1,
        `\refvar{CurveType}{}` type, const `\refvar{Normal3f}{}` *norm)
    : `\refvar[CurveCommon::type]{type}{}`(type), `\refvar{cpObj}{}`{c[0], c[1], c[2], c[3]},
      `\refvar[CurveCommon::width]{width}{}`{width0, width1} {
    if (norm) {
        `\refvar[CurveCommon::n]{n}{}`[0] = `\refvar{Normalize}{}`(norm[0]);
        `\refvar[CurveCommon::n]{n}{}`[1] = `\refvar{Normalize}{}`(norm[1]);
        `\refvar{normalAngle}{}` = std::acos(`\refvar{Clamp}{}`(`\refvar{Dot}{}`(`\refvar[CurveCommon::n]{n}{}`[0], `\refvar[CurveCommon::n]{n}{}`[1]), 0, 1));
        `\refvar{invSinNormalAngle}{}` = 1 / std::sin(`\refvar{normalAngle}{}`);
    }
}
\end{lstlisting}
\begin{lstlisting}
`\initcode{CurveCommon Declarations}{=}`
struct `\initvar{CurveCommon}{}` {
    const `\refvar{CurveType}{}` `\initvar[CurveCommon::type]{type}{}`;
    const `\refvar{Point3f}{}` `\initvar{cpObj}{}`[4];
    const `\refvar{Float}{}` `\initvar[CurveCommon::width]{width}{}`[2];
    `\refvar{Normal3f}{}` `\initvar[CurveCommon::n]{n}{}`[2];
    `\refvar{Float}{}` `\initvar{normalAngle}{}`, `\initvar{invSinNormalAngle}{}`;
};
\end{lstlisting}

\refvar{Curve}{}的边界框可通过利用\keyindex{凸包性质}{convex hull property}{convex hull凸包}计算,
这是贝塞尔曲线的性质,即它们一定位于其控制点的\keyindex{凸包}{convex hull}{}内
\sidenote{译者注:在实数向量空间中,对于给定集合$X$,
所有包含$X$的凸集的交集$S$称为$X$的凸包,
它可以用$X$内所有点$(x_1,\ldots,x_n)$的线性组合来构造,
即$\displaystyle S=\{\sum\limits_{j=1}^n{t_jx_j}|x_j\in X,\sum\limits_{j=1}^n{t_j}=1,t_j\in[0,1]\}$.}。
因此,控制点的边界框给出了底层曲线的保守边界。
方法\refvar[Curve::ObjectBound]{ObjectBound}{()}首先
计算1D贝塞尔段的控制点边界框来沿曲线的中心包围样条。
然后这些边界在曲线参数范围内被展开所取最大宽度的一半,
得到\refvar{Curve}{}表示的\refvar{Shape}{}的3D边界。
\begin{lstlisting}
`\refcode{Curve Method Definitions}{+=}\lastnext{CurveMethodDefinitions}`
`\refvar{Bounds3f}{}` `\initvar{Curve::ObjectBound}{}`() const {
    `\refcode{Compute object-space control points for curve segment, cpObj}{}`
    `\refvar{Bounds3f}{}` b = `\refvar[Union2]{Union}{}`(`\refvar{Bounds3f}{}`(cpObj[0], cpObj[1]),
                       `\refvar{Bounds3f}{}`(cpObj[2], cpObj[3]));
    `\refvar{Float}{}` width[2] = { `\refvar{Lerp}{}`(`\refvar{uMin}{}`, `\refvar{common}{}`->`\refvar[CurveCommon::width]{width}{}`[0], `\refvar{common}{}`->`\refvar[CurveCommon::width]{width}{}`[1]),
                       `\refvar{Lerp}{}`(`\refvar{uMax}{}`, `\refvar{common}{}`->`\refvar[CurveCommon::width]{width}{}`[0], `\refvar{common}{}`->`\refvar[CurveCommon::width]{width}{}`[1]) };
    return `\refvar{Expand}{}`(b, std::max(width[0], width[1]) * 0.5f);
}
\end{lstlisting}

类\refvar{CurveCommon}{}存储了整条曲线的控制点,
但\refvar{Curve}{}实例通常需要对应其$u$范围的
表示贝塞尔曲线的四个控制点。
这些控制点用称为\keyindex{开花}{blossoming}{}的技术计算。
三次贝塞尔样条簇
\sidenote{译者注:原文blossom。
通过繁琐但没有技巧性的多项式代换可以证明:
对于由控制点集$A=\{\bm p_i\}_{i=0}^3$定义的
三次贝塞尔样条簇$\bm p_A(u_0,u_1,u_2)$,
其中任意截取$u_j\in[u_j^0,u_j^1],j=0,1,2$,
并设$u_j^k(i)=\left\{\begin{array}{l}
        u_j^0,\ \text{若}i<k, \\
        u_j^1,\ \text{其他},
    \end{array}\right.$
然后取新控制点集$B=\{\bm q_i\}_{i=0}^3$,
其中$\bm q_i=\bm p(u_0^3(i),u_1^2(i),u_2^1(i))$,
并设新参数变量$\displaystyle v_j=\frac{u_j-u_j^0}{u_j^1-u_j^0}$,
则有$\bm p_B(v_0,v_1,v_2)\equiv\bm p_A(u_0,u_1,u_2)$,
即实现控制点集$A$到$B$的等价变换。}
$\bm p(u_0,u_1,u_2)$由三个阶段的线性插值定义,
从原始控制点开始:
\begin{align}\label{eq:3.4}
    \bm a_i & =(1-u_0)\bm p_i+u_0\bm p_{i+1},\quad i\in[0,1,2]\nonumber\, , \\
    \bm b_j & =(1-u_1)\bm a_j+u_1\bm a_{j+1},\quad j\in[0,1]\nonumber\, ,   \\
    \bm c   & =(1-u_2)\bm b_0+u_2\bm b_1\, .
\end{align}

簇$\bm p(u,u,u)$给出了在位置$u$处的曲线值
(要自己验证的话,用$u_i=u$展开\refeq{3.4},化简,并和\refeq{3.3}比较)。

\refvar{BlossomBezier}{()}实现了该计算。
\begin{lstlisting}
`\initcode{Curve Utility Functions}{=}\initnext{CurveUtilityFunctions}`
static `\refvar{Point3f}{}` `\initvar{BlossomBezier}{}`(const `\refvar{Point3f}{}` p[4], `\refvar{Float}{}` u0, `\refvar{Float}{}` u1,
        `\refvar{Float}{}` u2) {
    `\refvar{Point3f}{}` a[3] = { `\refvar{Lerp}{}`(u0, p[0], p[1]),
                     `\refvar{Lerp}{}`(u0, p[1], p[2]),
                     `\refvar{Lerp}{}`(u0, p[2], p[3]) };
    `\refvar{Point3f}{}` b[2] = { `\refvar{Lerp}{}`(u1, a[0], a[1]), `\refvar{Lerp}{}`(u1, a[1], a[2]) };
    return `\refvar{Lerp}{}`(u2, b[0], b[1]);
}
\end{lstlisting}
\begin{figure}[htbp]
    \centering\input{Pictures/chap03/blossom-pts.tex}
    \caption{一段贝塞尔曲线开花求控制点。
        \protect\refeq{3.5}中四簇给出从$u_{\min}$到$u_{\max}$的曲线控制点。
        开花提供了优雅方法计算整条曲线的子集曲线的贝塞尔控制点。}
    \label{fig:3.19}
\end{figure}

范围从$u_{\min}$到$u_{\max}$的曲线段的四个控制点由簇给出(\reffig{3.19}):
\begin{align}\label{eq:3.5}
    \bm p_0 & =\bm p(u_{\min},u_{\min},u_{\min})\nonumber\, , \\
    \bm p_1 & =\bm p(u_{\min},u_{\min},u_{\max})\nonumber\, , \\
    \bm p_2 & =\bm p(u_{\min},u_{\max},u_{\max})\nonumber\, , \\
    \bm p_3 & =\bm p(u_{\max},u_{\max},u_{\max})\, .
\end{align}

有了该机制,计算\refvar{Curve}{}负责的曲线段的四个控制点很简单。
\begin{lstlisting}
`\initcode{Compute object-space control points for curve segment, cpObj}{=}`
`\refvar{Point3f}{}` cpObj[4];
cpObj[0] = `\refvar{BlossomBezier}{}`(`\refvar{common}{}`->`\refvar{cpObj}{}`, `\refvar{uMin}{}`, `\refvar{uMin}{}`, `\refvar{uMin}{}`);
cpObj[1] = `\refvar{BlossomBezier}{}`(`\refvar{common}{}`->`\refvar{cpObj}{}`, `\refvar{uMin}{}`, `\refvar{uMin}{}`, `\refvar{uMax}{}`);
cpObj[2] = `\refvar{BlossomBezier}{}`(`\refvar{common}{}`->`\refvar{cpObj}{}`, `\refvar{uMin}{}`, `\refvar{uMax}{}`, `\refvar{uMax}{}`);
cpObj[3] = `\refvar{BlossomBezier}{}`(`\refvar{common}{}`->`\refvar{cpObj}{}`, `\refvar{uMax}{}`, `\refvar{uMax}{}`, `\refvar{uMax}{}`);
\end{lstlisting}

\refvar{Curve}{}相交算法基于尽快抛弃可以确定一定不会与光线相交的曲线段,
否则就递归地将曲线对半分为更小的两个曲线段然后再测试。
最终,曲线被线性近似以进行高效相交测试。
在一些准备工作后,\refvar{recursiveIntersect}{()}调用
使用\refvar{Curve}{}表示的整段曲线启动该进程。
\begin{lstlisting}
`\refcode{Curve Method Definitions}{+=}\lastnext{CurveMethodDefinitions}`
bool `\refvar{Curve}{}`::`\initvar[Curve::Intersect]{\refvar[Shape::Intersect]{Intersect}{}}{}`(const `\refvar{Ray}{}` &r, `\refvar{Float}{}` *tHit,
        `\refvar{SurfaceInteraction}{}` *isect, bool testAlphaTexture) const {
    `\refcode{Transform Ray to object space}{}`
    `\refcode{Compute object-space control points for curve segment, cpObj}{}`
    `\refcode{Project curve control points to plane perpendicular to ray}{}`
    `\refcode{Compute refinement depth for curve, maxDepth}{}`
    return `\refvar{recursiveIntersect}{}`(ray, tHit, isect, cp, `\refvar[Transform::Inverse]{Inverse}{}`(objectToRay),
                              `\refvar{uMin}{}`, `\refvar{uMax}{}`, maxDepth);
}
\end{lstlisting}

如同\refsub{三角形相交}的光线-三角形相交算法,
光线-曲线相交测试基于把曲线变换到射线端点在原点、
射线方向对齐$+z$轴的坐标系。
一开始就执行该变换极大减少了相交测试必须执行的运算量。

对于\refvar{Curve}{}形状,我们需要变换的显式表示,
所以这里用函数\refvar{LookAt}{()}来生成它。
原点是射线的端点,“注视”点是从原点沿射线方向偏移的点,
“向上”方向是正交于射线方向的任意方向。
\begin{lstlisting}
`\initcode{Project curve control points to plane perpendicular to ray}{=}`
`\refvar{Vector3f}{}` dx, dy;
`\refvar{CoordinateSystem}{}`(ray.`\refvar[Ray::d]{d}{}`, &dx, &dy);
`\refvar{Transform}{}` objectToRay = `\refvar{LookAt}{}`(ray.`\refvar[Ray::o]{o}{}`, ray.`\refvar[Ray::o]{o}{}` + ray.`\refvar[Ray::d]{d}{}`, dx);
`\refvar{Point3f}{}` cp[4] = { objectToRay(cpObj[0]), objectToRay(cpObj[1]),
                  objectToRay(cpObj[2]), objectToRay(cpObj[3]) };
\end{lstlisting}

计算细分曲线的最大次数,这样与最精细级别的最终线性化曲线的距离
就被限定在小于某微小固定的距离内。
我们不深入该计算的细节
\sidenote{译者注:我对这段代码的理解是:
通过比较{\ttfamily cp[1]}与{\ttfamily cp[0]、cp[2]}的中点之差以及
{\ttfamily cp[2]}与{\ttfamily cp[1]、cp[3]}的中点之差,并取全部维度的最大绝对值,
{\ttfamily L0}表示了曲线的弯曲程度;{\ttfamily eps}表示最大宽度的一半。
{\ttfamily L0}与{\ttfamily eps}的比值反映了相应精度要求,
取对数后对应最大划分次数。},
它在代码片\refcode{Compute refinement depth for curve, maxDepth}{}中实现。
\begin{lstlisting}
`\initcode{Compute refinement depth for curve, maxDepth}{=}`
`\refvar{Float}{}` L0 = 0;
for (int i = 0; i < 2; ++i)
    L0 = std::max(L0,
         std::max(std::max(std::abs(cp[i].x - 2 * cp[i + 1].x + cp[i + 2].x),
                           std::abs(cp[i].y - 2 * cp[i + 1].y + cp[i + 2].y)),
                  std::abs(cp[i].z - 2 * cp[i + 1].z + cp[i + 2].z)));
`\refvar{Float}{}` eps = std::max(`\refvar{common}{}`->`\refvar[CurveCommon::width]{width}{}`[0], `\refvar{common}{}`->`\refvar[CurveCommon::width]{width}{}`[1]) * .05f; // width / 20
#define LOG4(x) (std::log(x) * 0.7213475108f)
`\refvar{Float}{}` fr0 = LOG4(1.41421356237f * 12.f * L0 / (8.f * eps));
#undef LOG4
int r0 = (int)std::round(fr0);
int maxDepth = `\refvar{Clamp}{}`(r0, 0, 10);
\end{lstlisting}

然后方法\refvar{recursiveIntersect}{()}测试给定射线是否与
在给定参数范围{\ttfamily [u0,u1]}内的曲线段相交。
\begin{lstlisting}
`\refcode{Curve Method Definitions}{+=}\lastcode{CurveMethodDefinitions}`
bool `\refvar{Curve}{}`::`\initvar{recursiveIntersect}{}`(const `\refvar{Ray}{}` &ray, `\refvar{Float}{}` *tHit,
        `\refvar{SurfaceInteraction}{}` *isect, const `\refvar{Point3f}{}` cp[4],
        const `\refvar{Transform}{}` &rayToObject, `\refvar{Float}{}` u0, `\refvar{Float}{}` u1,
        int depth) const {
    `\refcode{Try to cull curve segment versus ray}{}`
    if (depth > 0) {
        `\refcode{Split curve segment into sub-segments and test for intersection}{}`
    } else {
        `\refcode{Intersect ray with curve segment}{}`
    }
}
\end{lstlisting}

该方法先开始检查射线是否与曲线段的边界框相交;
如果没有,则不可能相交,它就立即返回。
\begin{lstlisting}
`\initcode{Try to cull curve segment versus ray}{=}`
`\refcode{Compute bounding box of curve segment, curveBounds}{}`
`\refcode{Compute bounding box of ray, rayBounds}{}`
if (`\refvar{Overlaps}{}`(curveBounds, rayBounds) == false)
    return false;
\end{lstlisting}

沿着\refvar{Curve::ObjectBound}{()}的实现路线,
可以通过接收曲线控制点的边界并在曲线考虑的$u$范围内
展开最大宽度的一半来求得该段的保守边界框。
\begin{lstlisting}
`\initcode{Compute bounding box of curve segment, curveBounds}{=}`
`\refvar{Bounds3f}{}` curveBounds =
    `\refvar[Union2]{Union}{}`(`\refvar{Bounds3f}{}`(cp[0], cp[1]), `\refvar{Bounds3f}{}`(cp[2], cp[3]));
`\refvar{Float}{}` maxWidth = std::max(`\refvar{Lerp}{}`(u0, `\refvar{common}{}`->`\refvar[CurveCommon::width]{width}{}`[0], `\refvar{common}{}`->`\refvar[CurveCommon::width]{width}{}`[1]),
                          `\refvar{Lerp}{}`(u1, `\refvar{common}{}`->`\refvar[CurveCommon::width]{width}{}`[0], `\refvar{common}{}`->`\refvar[CurveCommon::width]{width}{}`[1]));
curveBounds = `\refvar{Expand}{}`(curveBounds, 0.5 * maxWidth);
\end{lstlisting}

因为在相交空间内射线端点在$(0,0,0)$且其方向对齐$+z$轴,
所以其边界框只包含$x$和$y$原点(\reffig{3.20});
其$z$范围由参数范围覆盖的相应部分给出。
\begin{figure}[htbp]
    \centering\input{Pictures/chap03/ray-origin-vs-curve.tex}
    \caption{光线-曲线边界测试。在光线坐标系中,射线的端点为$(0,0,0)$且
        其方向对齐$+z$轴。因此,如果2D点$(x,y)=(0,0)$在曲线段边界框外,
        则光线不可能与曲线相交。}
    \label{fig:3.20}
\end{figure}
\begin{lstlisting}
`\initcode{Compute bounding box of ray, rayBounds}{=}` 
`\refvar{Float}{}` rayLength = ray.`\refvar[Ray::d]{d}{}`.`\refvar{Length}{}`();
`\refvar{Float}{}` zMax = rayLength * ray.`\refvar{tMax}{}`;
`\refvar{Bounds3f}{}` rayBounds(`\refvar{Point3f}{}`(0, 0, 0), `\refvar{Point3f}{}`(0, 0, zMax));
\end{lstlisting}

如果光线与曲线的边界框相交且递归划分尚未触底,
则曲线的参数$u$范围划分为两半。
\refvar{SubdivideBezier}{()}计算七个控制点:
前四个对应第一半划分曲线的四个控制点,
后四个(从第一半的最后一个控制点开始)对应第二半的控制点。
调用两次\refvar{recursiveIntersect}{()}测试两条子段。
\begin{lstlisting}
`\initcode{Split curve segment into sub-segments and test for intersection}{=}`
`\refvar{Float}{}` uMid = 0.5f * (u0 + u1);
`\refvar{Point3f}{}` cpSplit[7];
`\refvar{SubdivideBezier}{}`(cp, cpSplit);
return (`\refvar{recursiveIntersect}{}`(ray, tHit, isect, &cpSplit[0], rayToObject,
                           u0, uMid, depth - 1) ||
        `\refvar{recursiveIntersect}{}`(ray, tHit, isect, &cpSplit[3], rayToObject,
                           uMid, u1, depth - 1));  
\end{lstlisting}

尽管我们可以用函数\refvar{BlossomBezier}{()}计算划分的曲线的控制点,
但可以利用我们总是恰好在参数范围的中点处划分曲线的事实更高效地计算它们。
函数\refvar{SubdivideBezier}{()}实现了该计算;
其计算的七个控制点对应使用$(0,0,0),(0,0,\frac{1}{2}),
    (0,\frac{1}{2},\frac{1}{2}),(\frac{1}{2},\frac{1}{2},\frac{1}{2}),
    (\frac{1}{2},\frac{1}{2},1),(\frac{1}{2},1,1)$和$(1,1,1)$作为\refeq{3.4}中的簇。
\begin{lstlisting}
`\refcode{Curve Utility Functions}{+=}\lastnext{CurveUtilityFunctions}`
inline void `\initvar{SubdivideBezier}{}`(const `\refvar{Point3f}{}` cp[4], `\refvar{Point3f}{}` cpSplit[7]) {
    cpSplit[0] = cp[0];
    cpSplit[1] = (cp[0] + cp[1]) / 2;
    cpSplit[2] = (cp[0] + 2 * cp[1] + cp[2]) / 4;
    cpSplit[3] = (cp[0] + 3 * cp[1] + 3 * cp[2] + cp[3]) / 8;
    cpSplit[4] = (cp[1] + 2 * cp[2] + cp[3]) / 4;
    cpSplit[5] = (cp[2] + cp[3]) / 2;
    cpSplit[6] = cp[3];
} 
\end{lstlisting}

在多次划分后,执行相交测试。
通过使用曲线的线性近似,
该测试的一部分得以更高效执行;
变差缩减性允许我们做该近似而不引入太大误差。
\begin{lstlisting}
`\initcode{Intersect ray with curve segment}{=}`
`\refcode{Test ray against segment endpoint boundaries}{}`
`\refcode{Compute line w that gives minimum distance to sample point}{}`
`\refcode{Compute u coordinate of curve intersection point and hitWidth}{}`
`\refcode{Test intersection point against curve width}{}`
`\refcode{Compute v coordinate of curve intersection point}{}`
`\refcode{Compute hit t and partial derivatives for curve intersection}{}`
return true;
\end{lstlisting}
\begin{figure}[htbp]
    \centering\input{Pictures/chap03/curve-segment-edges.tex}
    \caption{曲线段边界。一段大曲线的相交测试
        对垂直于段终点的直线(虚线)计算边函数。
        如果潜在交点(实心点)与曲线段在该边的不同侧,则被拒绝;
        应换为另一曲线段(如果出现在那侧)考虑该相交。}
    \label{fig:3.21}
\end{figure}

重要的是,相交测试只接受位于当前考虑的那段\refvar{Curve}{}曲面上的相交处。
因此,相交测试第一步是为垂直于曲线起点和终点的直线计算边函数并
对与之可能相交的点分类(\reffig{3.21})。
\begin{lstlisting}
`\initcode{Test ray against segment endpoint boundaries}{=}`
`\refcode{Test sample point against tangent perpendicular at curve start}{}`
`\refcode{Test sample point against tangent perpendicular at curve end}{}`
\end{lstlisting}

将曲线控制点投影到射线坐标系有两个原因让该测试更高效。
第一,因为射线方向朝向$+z$轴,问题简化为$x$和$y$的2D测试。
第二,因为射线端点在坐标系原点,我们需要分类的点是$(0,0)$,
这简化了边函数求值,就像光线-三角形相交测试那样。

边函数已在光线-三角形相交测试的\refeq{3.1}介绍了;也可见\reffig{3.14}。
为了定义边函数,我们需要垂直于曲线且过其起点的直线上任意的两点。
第一个控制点$\bm p_0$是第一个点的不错选择。
对于第二个点,我们将计算垂直于曲线切线的向量并将该偏移量加到控制点上。

\refeq{3.3}的微分说明曲线在第一个控制点$\bm p_0$处的切线为$3(\bm p_1-\bm p_0)$.
缩放因子在这里无关紧要,所以我们这里用$\bm t=\bm p_1-\bm p_0$.
在2D中计算垂直于切线的向量很容易:只需要交换$x$和$y$坐标并让其中一个取相反数。
(为了理解为什么这样可行,考虑点积$(x,y)\cdot(y,-x)=xy+(-yx)=0$.
因为两个向量夹角的余弦为零,它们一定垂直。)
因此,边上的第二个点为
\begin{align*}
    \bm p_0+(p_{1y}-p_{0y},-(p_{1x}-p_{0x}))=\bm p_0+(p_{1y}-p_{0y},p_{0x}-p_{1x})\, .
\end{align*}
将这两点代入边函数的定义\refeq{3.1},化简得
\begin{align*}
    e(\bm p)=(p_{1y}-p_{0y})(p_y-p_{0y})-(p_x-p_{0x})(p_{0x}-p_{1x})\, .
\end{align*}
最后,代入$\bm p=(0,0)$得到最后测试的表达式:
\begin{align*}
    e((0,0))=(p_{1y}-p_{0y})(-p_{0y})+p_{0x}(p_{0x}-p_{1x})\, .
\end{align*}
\begin{lstlisting}
`\initcode{Test sample point against tangent perpendicular at curve start}{=}`
`\refvar{Float}{}` edge = (cp[1].y - cp[0].y) * -cp[0].y +
              cp[0].x * (cp[0].x - cp[1].x);
if (edge < 0)
    return false;
\end{lstlisting}

代码片\refcode{Test sample point against tangent perpendicular at curve end}{}则
对曲线终点做相应测试。
\begin{lstlisting}
`\initcode{Test sample point against tangent perpendicular at curve end}{=}`
edge = (cp[2].y - cp[3].y) * -cp[3].y +
        cp[3].x * (cp[3].x - cp[2].x);
if (edge < 0)
    return false;
\end{lstlisting}

测试的下一部分是确定点$(0,0)$离曲线最近处沿曲线段的$u$值。
如果到那点中心的距离不超过曲线宽度,则这就是交点。
对三次贝塞尔曲线确定该距离并不高效,
所以该相交方法替换为用线段近似曲线来计算该$u$值。
\begin{figure}[htbp]
    \centering\input{Pictures/chap03/curve-approx-linear.tex}
    \caption{用线段近似三次贝塞尔曲线。
        对于这部分光线-曲线相交测试,我们用过其起点和终点的线段(虚线)
        近似三次贝塞尔曲线(实际中,被划分后,曲线几乎已经是线性的了,
        所以误差比该图中画的更小)。}
    \label{fig:3.22}
\end{figure}

我们将用从其起点$\bm p_0$到其终点$\bm p_3$的
由$u$参数化的线段线性近似贝塞尔曲线。
这时,$w=0$处的位置为$\bm p_0$,$w=1$处为$\bm p_3$(\reffig{3.22})。
我们现在的任务是计算直线上离点$\bm p$最近的$\bm p'$沿直线的$u$值。
运用的关键是,在$\bm p'$处,从直线上相应点
到$\bm p$的向量将垂直于该直线(\reffig{3.23}(a))。
\begin{figure}[htbp]
    \centering\includegraphics[width=0.4\linewidth]{chap03/Pointlinedistance.eps}
    \caption{(a)给定无限直线和点$\bm p$,则从该点到直线上
        最近点$\bm p'$的向量垂直于该直线。
        (b)因为该向量垂直,我们可以计算从直线上第一点到
        最近点$\bm p'$的距离为$d=\|\bm p-\bm p_0\|\cos\theta$.}
    \label{fig:3.23}
\end{figure}

\refeq{2.1}给出了两向量点积、它们的长度以及夹角余弦的关系。
特别地,它向我们展示了怎样计算从$\bm p_0$到$\bm p$的向量
与从$\bm p_0$到$\bm p_3$的向量之间夹角的余弦:
\begin{align*}
    \cos\theta=\frac{(\bm p-\bm p_0)\cdot(\bm p_3-\bm p_0)}{\|\bm p-\bm p_0\|\|\bm p_3-\bm p_0\|}\, .
\end{align*}

因为从$\bm p'$到$\bm p$的向量垂直于该直线
(\reffig{3.23}(b)\sidenote{译者注:原书中$d$的位置标错了,已改正。}),
则我们可以计算沿直线从$\bm p_0$到$\bm p'$的距离为
\begin{align*}
    d=\|\bm p-\bm p_0\|\cos\theta=\frac{(\bm p-\bm p_0)\cdot(\bm p_3-\bm p_0)}{\|\bm p_3-\bm p_0\|}\, .
\end{align*}

最后,沿直线的参数偏移量$w$是$d$与直线长度的比值,
\begin{align*}
    w=\frac{d}{\|\bm p_3-\bm p_0\|}=\frac{(\bm p-\bm p_0)\cdot(\bm p_3-\bm p_0)}{\|\bm p_3-\bm p_0\|^2}\, .
\end{align*}

因为事实上相交坐标系中$\bm p=(0,0)$,所以$w$值的计算最后有所简化。
\begin{lstlisting}
`\initcode{Compute line w that gives minimum distance to sample point}{=}`
`\refvar{Vector2f}{}` segmentDirection = `\refvar{Point2f}{}`(cp[3]) - `\refvar{Point2f}{}`(cp[0]);
`\refvar{Float}{}` denom = segmentDirection.`\refvar{LengthSquared}{}`();
if (denom == 0)
    return false;
`\refvar{Float}{}` w = `\refvar{Dot}{}`(-`\refvar{Vector2f}{}`(cp[0]), segmentDirection) / denom;
\end{lstlisting}

贝塞尔曲线上到候选交点最近(假定)点的参数$u$坐标通过
沿该段$u$的范围线性插值算得。
给定该值,就能计算曲线在该点处的宽度。
\begin{lstlisting}
`\initcode{Compute u coordinate of curve intersection point and hitWidth}{=}`
`\refvar{Float}{}` u = `\refvar{Clamp}{}`(`\refvar{Lerp}{}`(w, u0, u1), u0, u1);
`\refvar{Float}{}` hitWidth = `\refvar{Lerp}{}`(u, `\refvar{common}{}`->`\refvar[CurveCommon::width]{width}{}`[0], `\refvar{common}{}`->`\refvar[CurveCommon::width]{width}{}`[1]);
`\refvar{Normal3f}{}` nHit;
if (`\refvar{common}{}`->`\refvar[CurveCommon::type]{type}{}` == `\refvar{CurveType}{}`::`\refvar{Ribbon}{}`) {
    `\refcode{Scale hitWidth based on ribbon orientation}{}`
}
\end{lstlisting}

对于\refvar{Ribbon}{}曲线,曲线并不总是朝向光线。
相反,它的朝向是在于每个端点处给定的曲面法线之间插值的。
这里,球面线性插值用于插值$u$处的法线(回想\refsub{四元数插值})。
再用规范化的射线方向与丝带朝向间的夹角余弦缩放曲线宽度,
这样它就反映了给定方向的曲线可见宽度。
\begin{lstlisting}
`\initcode{Scale hitWidth based on ribbon orientation}{=}`
`\refvar{Float}{}` sin0 = std::sin((1 - u) * `\refvar{common}{}`->`\refvar{normalAngle}{}`) *
    `\refvar{common}{}`->`\refvar{invSinNormalAngle}{}`;
`\refvar{Float}{}` sin1 = std::sin(u * `\refvar{common}{}`->`\refvar{normalAngle}{}`) *
    `\refvar{common}{}`->`\refvar{invSinNormalAngle}{}`;
nHit = sin0 * `\refvar{common}{}`->`\refvar[CurveCommon::n]{n}{}`[0] + sin1 * `\refvar{common}{}`->`\refvar[CurveCommon::n]{n}{}`[1];
hitWidth *= `\refvar{AbsDot}{}`(nHit, ray.`\refvar[Ray::d]{d}{}`) / rayLength;
\end{lstlisting}

为了最终将可能的相交处分类为命中或未命中,
必须用函数\refvar{EvalBezier}{()}
求贝塞尔曲线在$u$处的值
(因为控制点{\ttfamily cp}表示当前考虑的曲线段,
然而既然$w$在范围$[0,1]$内,在函数调用中使用$w$而不是$u$就很重要)。
曲线在该点的导数很快也会有用,所以现在记录下来。

我们想测试从$\bm p$到曲线上该点{\ttfamily pc}的
距离是否小于曲线宽度的一半。
因为$\bm p=(0,0)$,我们可以等价地测试从{\ttfamily pc}到原点
的距离是否小于宽度的一半,或者等价地,
平方距离是否小于宽度平方的四分之一。
如果该测试通过,最后一件事是检查交点是否在射线的参数$t$范围内。
\begin{lstlisting}
`\initcode{Test intersection point against curve width}{=}`
`\refvar{Vector3f}{}` dpcdw;
`\refvar{Point3f}{}` pc = `\refvar{EvalBezier}{}`(cp, `\refvar{Clamp}{}`(w, 0, 1), &dpcdw);
`\refvar{Float}{}` ptCurveDist2 = pc.x * pc.x + pc.y * pc.y;
if (ptCurveDist2 > hitWidth * hitWidth * .25)
    return false;
if (pc.z < 0 || pc.z > zMax)
    return false;
\end{lstlisting}

\refvar{EvalBezier}{()}计算簇$\bm p(u,u,u)$来求贝塞尔样条上的一点。
它也可选地返回曲线在该点处的导数\sidenote{译者注:读者可以动手验算一下。}。
\begin{lstlisting}
`\refcode{Curve Utility Functions}{+=}\lastcode{CurveUtilityFunctions}`
static `\refvar{Point3f}{}` `\initvar{EvalBezier}{}`(const `\refvar{Point3f}{}` cp[4], `\refvar{Float}{}` u,
        `\refvar{Vector3f}{}` *deriv = nullptr) {
    `\refvar{Point3f}{}` cp1[3] = { `\refvar{Lerp}{}`(u, cp[0], cp[1]), `\refvar{Lerp}{}`(u, cp[1], cp[2]),
                       `\refvar{Lerp}{}`(u, cp[2], cp[3]) };
    `\refvar{Point3f}{}` cp2[2] = { `\refvar{Lerp}{}`(u, cp1[0], cp1[1]), `\refvar{Lerp}{}`(u, cp1[1], cp1[2]) };
    if (deriv)
        *deriv = (`\refvar{Float}{}`)3 * (cp2[1] - cp2[0]);
    return `\refvar{Lerp}{}`(u, cp2[0], cp2[1]);
}
\end{lstlisting}

如果之前的测试都通过了,我们就找到了有效的相交,
现在可以计算交点的$v$坐标了。
曲线的$v$坐标在0到1内变化,曲线中心取值为0.5;
这里我们根据过曲线上的点{\ttfamily pc}和沿其导数的点的边函数
将交点$(0,0)$分类,决定交点在中心的哪边以及该怎样计算$v$.
\begin{lstlisting}
`\initcode{Compute v coordinate of curve intersection point}{=}`
`\refvar{Float}{}` ptCurveDist = std::sqrt(ptCurveDist2);
`\refvar{Float}{}` edgeFunc = dpcdw.x * -pc.y + pc.x * dpcdw.y;
`\refvar{Float}{}` v = (edgeFunc > 0) ? 0.5f + ptCurveDist / hitWidth :
                           0.5f - ptCurveDist / hitWidth;
\end{lstlisting}

最后,计算偏导数,相交处的\refvar{SurfaceInteraction}{}可以初始化了。
\begin{lstlisting}
`\initcode{Compute hit t and partial derivatives for curve intersection}{=}`
if (tHit != nullptr) {
    *tHit = pc.z / rayLength;
    `\refcode{Compute error bounds for curve intersection}{}`
    `\refcode{Compute $\partial$p/$\partial$u and $\partial$p/$\partial$v for curve intersection}{}`
    *isect = (*`\refvar{ObjectToWorld}{}`)(`\refvar{SurfaceInteraction}{}`(
        ray(pc.z), pError, `\refvar{Point2f}{}`(u, v), -ray.`\refvar[Ray::d]{d}{}`, dpdu, dpdv,
        `\refvar{Normal3f}{}`(0, 0, 0), `\refvar{Normal3f}{}`(0, 0, 0), ray.`\refvar[Ray::time]{time}{}`, this));
}
\end{lstlisting}

偏导数$\displaystyle\frac{\partial \bm p}{\partial u}$直接
来自于底层贝塞尔曲线的导数。
第二个偏导数$\displaystyle\frac{\partial \bm p}{\partial v}$则
基于曲线类型用不同方式计算。
对于丝带,我们有$\displaystyle\frac{\partial \bm p}{\partial u}$和
曲面法线,因此$\displaystyle\frac{\partial \bm p}{\partial v}$必须
是使得$\displaystyle\frac{\partial \bm p}{\partial u}\times\frac{\partial \bm p}{\partial v}=\bm n$的向量
且长度等于曲线宽度。
\begin{lstlisting}
`\initcode{Compute $\partial$p/$\partial$u and $\partial$p/$\partial$v for curve intersection}{=}`
`\refvar{Vector3f}{}` dpdu, dpdv;
`\refvar{EvalBezier}{}`(`\refvar{common}{}`->`\refvar{cpObj}{}`, u, &dpdu);
if (`\refvar{common}{}`->`\refvar[CurveCommon::type]{type}{}` == `\refvar{CurveType}{}`::`\refvar{Ribbon}{}`)
    dpdv = `\refvar{Normalize}{}`(`\refvar{Cross}{}`(nHit, dpdu)) * hitWidth;
else {
    `\refcode{Compute curve $\partial$p/$\partial$v for flat and cylinder curves}{}`
}
\end{lstlisting}

对于平坦和圆柱曲线,我们变换$\displaystyle\frac{\partial \bm p}{\partial u}$到相交坐标系。
对于平坦曲线,我们知道$\displaystyle\frac{\partial \bm p}{\partial v}$位于$xy$平面内、
垂直于$\displaystyle\frac{\partial \bm p}{\partial u}$且长度等于{\ttfamily hitWidth}。
我们可以用和之前求垂直于曲线段的边界边同样的方法求得2D垂直向量。
\begin{lstlisting}
`\initcode{Compute curve $\partial$p/$\partial$v for flat and cylinder curves}{=}`
`\refvar{Vector3f}{}` dpduPlane = (`\refvar[Transform::Inverse]{Inverse}{}`(rayToObject))(dpdu);
`\refvar{Vector3f}{}` dpdvPlane = `\refvar{Normalize}{}`(`\refvar{Vector3f}{}`(-dpduPlane.y, dpduPlane.x, 0)) *
                     hitWidth;
if (`\refvar{common}{}`->`\refvar[CurveCommon::type]{type}{}` == `\refvar{CurveType}{}`::`\refvar[CurveType::Cylinder]{Cylinder}{}`) {
    `\refcode{Rotate dpdvPlane to give cylindrical appearance}{}`
}
dpdv = rayToObject(dpdvPlane);
\end{lstlisting}

圆柱曲线的向量$\displaystyle\frac{\partial \bm p}{\partial v}$沿轴{\ttfamily dpduPlane}旋转
使其外观像圆柱体横截面
\sidenote{译者注:对于圆柱曲线,在\refcode{Compute v coordinate of curve intersection point}{}中
计算$v$以及此处计算$\theta$的方法已足够满足实际应用了。
但这样的线性插值是对“圆柱”特性的近似描述。
若要精确描述,则应是$v=\frac{1}{2}(1+\sin\theta)$,也即$\theta=\arcsin(2v-1)$(式里$\theta$为弧度制)。
然而三角函数的计算开销很大,带来的视觉差异又很小,所以得不偿失。
更多讨论详见\url{https://github.com/mmp/pbrt-v4/issues/200}。}。
\begin{lstlisting}
`\initcode{Rotate dpdvPlane to give cylindrical appearance}{=}`
`\refvar{Float}{}` theta = `\refvar{Lerp}{}`(v, -90., 90.);
`\refvar{Transform}{}` rot = `\refvar{Rotate}{}`(-theta, dpduPlane);
dpdvPlane = rot(dpdvPlane);
\end{lstlisting}

这里没有介绍的方法\refvar{Curve::Area}{()}首先
用其控制点连线长度近似曲线长度。
然后它再用该长度乘以其范围内宽度的均值以近似整个曲面面积。